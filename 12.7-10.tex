\documentclass[12pt]{article}
\usepackage{amsmath}
\usepackage{booktabs}  % Added booktabs for better table rules
\usepackage{circuitikz}

\newcommand{\myCode}{%
  \title{Physics Assignment}
  \author{Hiba Muhammed \\
          EE23BTECH11026}

  \begin{document}
    \maketitle

    \section*{Problem Statement}
    A radio can tune over the frequency range of a portion of the MW broadcast band: (800 kHz to 1200 kHz). If its LC circuit has an effective inductance (\(L\)) and a variable capacitor with capacitance (\(C\)), what must be the range of \(C\)?

    \section*{Solution}
    To find the range of the variable capacitor (\(C\)) for a radio tuning over the frequency range of the MW broadcast band with an effective inductance (\(L\)), we can use the formula for the resonant frequency (\(f\)) of an LC circuit:
    \[ f = \frac{1}{2\pi\sqrt{LC}} \]

    For this problem, we can rearrange the formula to solve for \(C\):
    \[ C = \frac{1}{(2\pi f)^2L} \]

    Given the frequency range of 800 kHz to 1200 kHz, we can find the range of \(C\) by substituting these values into the formula:
    \[ C_1 = \frac{1}{(2\pi \times 800 \times 10^3)^2 \times L} \approx 198.1 \, \text{pF} \]
    \[ C_2 = \frac{1}{(2\pi \times 1200 \times 10^3)^2 \times L} \approx 88.04 \, \text{pF} \]

    So, the variable capacitor should have a frequency range between 198.1pF and 88.04pF.

\begin{table}[h]
  \centering
  \caption{Input Parameters given in the question}
  \label{tab:parameters}
  \begin{tabular}{|c|c|c|}
    \hline
    \textbf{Symbol} & \textbf{Value} & \textbf{Description} \\
    \hline
    \(L\) & \(200 \mu H\) & Inductance of the circuit \\
    \hline
    \(f_{\text{min}}\) & 800 kHz & Minimum operating frequency \\
    \hline
    \(f_{\text{max}}\) & 1200 kHz & Maximum operating frequency \\
    \hline
    \(C_1\) &   & Maximum capacitance value \\
    \hline
    \(C_2\) &  & Minimum capacitance value \\
    \hline
  \end{tabular}
\end{table}

\begin{figure}[h]
  \centering
  \begin{circuitikz}
    \draw (0,0) to[V, v=\(V_{\text{in}}\)] (0,2) to[L, l=\(L\)] (2,2) to[C, l=\(C\)] (4,2) -- (4,0) -- (0,0);
  \end{circuitikz}
  \caption{LC Circuit Diagram with Voltage Source (Time Domain)}
  \label{fig:lc-circuit-time-domain-voltage}
\end{figure}

\begin{figure}[h]
  \centering
  \begin{circuitikz}
    \draw (0,0) to[V, v=\(V_{\text{in}}\)] (0,2) to[L, l=\(sL\)] (2,2) to[C, l=\(\frac{1}{sC}\)] (4,2) -- (4,0) -- (0,0);
  \end{circuitikz}
  \caption{LC Circuit Diagram with Voltage Source (s Domain)}
  \label{fig:lc-circuit-s-domain-voltage}
\end{figure}

\begin{circuitikz}
    % Antenna
    \draw (0,0) to[C, l=$C_{\mathrm{ant}}$] (2,0) to[L, l=$L_{\mathrm{ant}}$] (4,0) to[R, l=$R_{\mathrm{ant}}$] (6,0) -- (8,0);
    
    % Tuning circuit
    \draw (8,0) -- (8,-2) to[C, l=$C_{\mathrm{tune}}$] (6,-2) -- (6,0);
    
    % Amplifier
    \draw (8,0) -- (10,0) to[voltage source, l=$V_{\mathrm{RF}}$] (10,-2) -- (8,-2);
    
    % Mixer
    \draw (10,0) -- (12,0) node[mixer] {};
    
    % Intermediate Frequency (IF) stage
    \draw (12,0) -- (14,0) node[op amp] (opamp) {}
        (opamp.-) -- (14,-2) -- (12,-2) node[ground] {}
        (opamp.+) -- (10,-2) node[ground] {}
        (opamp.out) -- (16,0);
    
    % Demodulator
    \draw (16,0) -- (18,0) to[diode] (18,-2) -- (16,-2);
    
    % Audio amplifier
    \draw (18,0) -- (20,0) to[R, l=$R_{\mathrm{audio}}$] (20,-2) to[C, l=$C_{\mathrm{audio}}$] (20,-4) node[op amp] (audioamp) {};
    
    % Output
    \draw (audioamp.out) -- (22,0) node[right] {$V_{\mathrm{out}}$};
    
    % Ground
    \draw (0,0) -- (0,-4) node[ground] {};
\end{circuitikz}
  \end{document}
}
\myCode



