\documentclass[12pt]{article}
\usepackage{amsmath}
\usepackage{booktabs}
\usepackage{circuitikz}

\newcommand{\myCode}{%
  \title{Physics Assignment}
  \author{Hiba Muhammed \\
          EE23BTECH11026}

  \begin{document}
    \maketitle

    \section*{Problem Statement}
    A radio can tune over the frequency range of a portion of the MW broadcast band: (800 kHz to 1200 kHz). If its LC circuit has an effective inductance (\(L\)) and a variable capacitor with capacitance (\(C\)), what must be the range of \(C\)?

    \section*{Solution}
    To find the range of the variable capacitor (\(C\)) for a radio tuning over the frequency range of the MW broadcast band with an effective inductance (\(L\)), we can use the formula for the resonant frequency (\(f\)) of an LC circuit:
    \[ f = \frac{1}{2\pi\sqrt{LC}} \]

    For this problem, we can rearrange the formula to solve for \(C\):
    \[ C = \frac{1}{(2\pi f)^2L} \]

    Given the frequency range of 800 kHz to 1200 kHz, we can find the range of \(C\) by substituting these values into the formula:
    \[ C_1 = \frac{1}{(2\pi \times 800 \times 10^3)^2 \times L} \approx 198.1 \, \text{pF} \]
    \[ C_2 = \frac{1}{(2\pi \times 1200 \times 10^3)^2 \times L} \approx 88.04 \, \text{pF} \]

    So, the variable capacitor should have a frequency range between 198.1pF and 88.04pF.

    \begin{table}[h]
      \centering
      \caption{Input Parameters given in the question}
      \label{tab:parameters}
      \begin{tabular}{|c|c|c|}
        \hline
        \textbf{Symbol} & \textbf{Value} & \textbf{Description} \\
        \hline
        \(L\) & \(200 \mu H\) & Inductance of the circuit \\
        \(f_{\text{min}}\) & 800 kHz & Minimum operating frequency \\
        \(f_{\text{max}}\) & 1200 kHz & Maximum operating frequency \\
        \(C_1\) &  \(\frac{1}{(2\pi \times 800 \times 10^3)^2 \times 200 \times 10^{-6}} \approx 198.1 \, \text{pF}\) & Maximum capacitance value \\
        \(C_2\) & \(\frac{1}{(2\pi \times 1200 \times 10^3)^2 \times 200 \times 10^{-6}} \approx 88.04 \, \text{pF}\) & Minimum capacitance value \\
        \hline
      \end{tabular}
    \end{table}

    \begin{figure}[h]
      \centering
      \begin{circuitikz}
        \draw (0,0) to[L, l=\(L\)] (2,0) to[C, l=\(C\)] (4,0) -- (4,-1) -- (0,-1) -- (0,0);
      \end{circuitikz}
      \caption{LC Circuit Diagram (Time Domain)}
      \label{fig:lc-circuit-time-domain}
    \end{figure}

    \begin{figure}[h]
      \centering
      \begin{circuitikz}
        \draw (0,0) to[L, l=\(sL\)] (2,0) to[C, l=\(\frac{1}{sC}\)] (4,0) -- (4,-1) -- (0,-1) -- (0,0);
      \end{circuitikz}
      \caption{LC Circuit Diagram (s Domain)}
      \label{fig:lc-circuit-s-domain}
    \end{figure}
  \end{document}
}
\myCode


