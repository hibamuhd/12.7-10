\documentclass[12pt]{article}
\usepackage{amsmath}
\usepackage{booktabs}
\usepackage{circuitikz}

\begin{document}

\title{Physics Assignment}
\author{Hiba Muhammed \\
        EE23BTECH11026}
\maketitle

\section*{Problem Statement}
A radio can tune over the frequency range of a portion of MW broadcast band: (800kHz to 1200kHz). If its LC circuit has an effective inductance of \(200\mu H\), what must be the range of its variable capacitor?

\section*{Solution}
To find the range of the variable capacitor (\(C\)) for a radio tuning over the frequency range of the MW broadcast band with an effective inductance (\(L\)) of \(200\mu H\), we can use the formula for the resonant frequency (\(f\)) of an LC circuit:
\[ f = \frac{1}{2\pi\sqrt{LC}} \]

For this problem, we can rearrange the formula to solve for \(C\):
\[ C = \frac{1}{(2\pi f)^2L} \]

Given the frequency range of 800 kHz to 1200 kHz, we can find the range of \(C\) by substituting these values into the formula:
\[ C_1 = \frac{1}{(2\pi \times 800 \times 10^3)^2 \times 200 \times 10^{-6}} = 197.8 \, \text{pF} \]
\[ C_2 = \frac{1}{(2\pi \times 1200 \times 10^3)^2 \times 200 \times 10^{-6}} = 87.9 \, \text{pF} \]

So, the variable capacitor should have a frequency range between 87.9 pF to 197.8 pF.

\begin{table}[h]
  \centering
  \caption{Input Parameters given in question}
  \label{tab:parameters}
  \begin{tabular}{|c|c|}
    \hline
    \textbf{Parameter} & \textbf{Value} \\
    \hline
    Inductance (\(L\)) & \(200\mu H\) \\
    Minimum Frequency & 800 kHz \\
    Maximum Frequency & 1200 kHz \\
    \hline
  \end{tabular}
\end{table}

\begin{figure}[h]
  \centering
  \begin{circuitikz}
    \draw (0,0) to[L, l=\(L\)] (2,0) to[C, l=\(C\)] (4,0);
    \draw (0,0) -- (0,-1);
    \draw (4,0) -- (4,-1);
    \node at (2,-1) {\(f\)};
  \end{circuitikz}
  \caption{LC Circuit Diagram}
  \label{fig:lc-circuit}
\end{figure}

\end{document}

